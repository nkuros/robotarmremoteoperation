% ---
% Arquivo com as ideias iniciais do Trabalho de Conclusão de Curso do aluno
% Daniel Noriaki Kurosawa
% da Escola Politécnica da Universidade de São Paulo
% ---
% ---

\setlength{\absparsep}{18pt} % ajusta o espaçamento dos parágrafos do resumo
\chapter{Objetivo}\label{cap-objetivo}

Este estudo se propõe realizar o controle de um braço mecânico simplificando seu método de entrada de comandos sem elevar significativamente seu custo de projeto.
Colocado este objetivo, uma primeira preocupação do projeto é quanto a sua adaptabilidade. Para que o projeto possa ser aplicado a uma ampla gama de finalidades, o projeto foi subdividido em pequenos subprojetos, que podem ser utilizados com independentemente. Para isso será utilizado um ao realizar a comunicação em tempo real através da internet, introduzindo o conceito de Internet of  Things (IoT) ao projeto. 
Serão levadas em considerações parâmetros como a robustez, taxa de transmissão de dados, custo da transmissão de dados, segurança do meio de comunicação, sendo necessária uma análise usando Architecture Tradeoff Analysis Method (ATAM) .
Uma vez que o controle é realizado a distancia, este trabalho se propõe a implementar um sistema de câmeras imersivas estereoscópicas, em que a movimentação de duas câmeras instaladas próximas ao braço robótico será controlada pela movimentação da cabeça do usuário, sendo as imagens assim obtidas transmitidas uma à cada olho do usuário através de óculos visores, provocando a sensação de profundidade.
O projeto será realizado em conjunto com a empresa Scopus Soluções em TI, que custeará parcialmente o projeto e fornecerá  know-how disponível na empresa.