% ---
% Arquivo com os resumos do Trabalho de Conclusão de Curso do aluno
% Daniel Noriaki Kurosawa
% da Escola Politécnica da Universidade de São Paulo
% ---

% resumo em português
\setlength{\absparsep}{18pt} % ajusta o espaçamento dos parágrafos do resumo
\begin{resumo}
	Este trabalho estuda a interação imersiva em um ambiente remoto através do uso de estereoscopia e de operação ótica sem marcadores de um braço robótico.
	
	Para tal objetivo, foi definido um projeto cuja execução gerou um sistema composto de:
	\begin{itemize}
		\item Microsoft Kinect, que envia os dados capturados a um braço robótico através de um servidor.
		\item Um sistema de câmeras estereoscópicas que envia suas imagens a um servidor, que podem ser acessadas remotamente por um óculos de realidade virtual ou qualquer dispositivo que tenha um browser compatível instalado.
		\item Um sistema de captura de movimentos da cabeça do usuário via giroscópio, que envia estes dados remotamente a um conjunto de motores localizados na base da câmera, de modo a sincronizar os movimentos da cabeça do usuário a visão das câmeras.		
	\end{itemize}

	 Os resultados finais indicam um grande espaço para o desenvolvimento de projetos deste tipo em aplicações futuras, apesar da necessidade de melhora de precisão para usos comerciais.
	
	\textbf{Palavras-chave}: Internet of Things. Microsoft Kinect. Estereoscopia. Braço Robótico. IoT. 
\end{resumo}

% resumo em inglês
\begin{resumo}[Abstract]
	\begin{otherlanguage*}{english}
		This work studies the immersive interaction with a remote enviroment by using stereoscopy and markerless operation of a robotic arm.


Para tal objetivo, foi definido um projeto cuja execução gerou um sistema composto de:
which create a system that comprises:
\begin{itemize}
	\item A Microsoft Kinect, which sends captured data to a robot controller via server.
	\item A stereoscopic camera system, which sends its images to a server, which are in turn accessed by VR Glasses or any device with a compatible browser installed.
	\item A gyroscope based head-tracking system, which sends user's head pan/tilt data to motors located in the camera system's base, allowing the cameras to move in synchronization with the user's head movements.		
\end{itemize}
Final results indicate great applicability for this approach in future uses. In it's current state however, this approach lacks precision necessary for commercial use.
	
		\vspace{\onelineskip}
		
		\noindent 
		\textbf{Keywords}: Internet of Things. Microsoft Kinect. Stereoscopy.  Robotic Arm. IoT. 
	\end{otherlanguage*}
\end{resumo}