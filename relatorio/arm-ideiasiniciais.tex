% ---
% Arquivo com as ideias iniciais do Trabalho de Conclusão de Curso do aluno
% Daniel Noriaki Kurosawa
% da Escola Politécnica da Universidade de São Paulo
% ---
% ---

	\setlength{\absparsep}{18pt} % ajusta o espaçamento dos parágrafos do resumo
	\chapter{Ideias iniciais}\label{cap-ideiasiniciais}
	
	Neste capítulo, expõe-se as principais abordagens e ideias levantadas durante a
	fase inicial do projeto, bem como o processo de decisão adotado para a escolha da ideia final. Analogamente à abordagem iterativa de desenvolvimento adotada durante o projeto, cada subcapítulo será uma adição ao anterior.\par
	
	\section{Ideias iniciais e alternativas do projeto}\label{subsec-iniciais}

	As ideias iniciais continham premissas que envolviam o uso de redes neurais para o controle do braço mecânico, utilizando um controlador ANFIS-PID para o controle dos motores.
Posteriormente, em virtude de limitações do hardware e a complexidade adicionada ao projeto, os algoritmos de redes neurais foram excluídos do escopo do projeto. Assim, a pesquisa inicial tratou os seguintes temas:\par


\begin{itemize}[noitemsep]
	\item Geração de trajetória para o braço usando Microsoft Kinect;
	\item Captura de movimentos usando giroscópio;
	\item Algoritmos anti-drifting de giroscópio;
	\item Captura de imagens estereoscópicas em tempo real;
	\item Arduino;
	\item Raspberry Pi;
	\item Particle Photon;
	\item Internet of Things;
\end{itemize}

\section{Formato final}\label{subsec-formatofinal}

