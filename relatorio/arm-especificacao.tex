% ---
% Arquivo com as especificações do Trabalho de Conclusão de Curso do aluno
% Daniel Noriaki Kurosawa
% da Escola Politécnica da Universidade de São Paulo
% ---
	% ---
		\chapter{Especificacões}\label{cap-especificacao}
		
		
	Esta seção baseia-se nos 5 pontos fundamentais da ODP (Open Distributed Processing)
	para especificar o projeto a ser desenvolvido. Assim, divide-se este capítulo em:
	Visão Empresarial, de Informação, Computacional, de Engenharia/Infraestrutura e de
	Tecnologia.
		\section{Visão Empresarial}\label{sec-empresarial}
	\subsection{Monetização do projeto}
	O projeto a ser desenvolvido tem o intuito de analisar o comportamento do motorista
	ao dirigir, podendo assim ser utilizado como uma ferramenta de grande auxílio para
	companhias de seguro e monitoramento de frota.
	Companhias de seguros poderiam utilizar o sistema proposto para implementar
	um modelo de seguros para automóveis baseado em uma cobrança variável, segundo os
	hábitos de direção do motorista. Se um cliente é bastante cuidadoso no volante, o preço
	do seguro de seu carro seria menor. Contudo, seria necessário realizar algumas melhorias
	no projeto, como por exemplo:
	Impedir que o usuário desligue o módulo com intuito de mascarar os dados captados.
	Sem esta funcionalidade, um usuário mal-intencionado poderia desligar o aparelho
	quando estiver dirigindo de maneira perigosa;
	Enviar os dados sem o uso do celular do usuário, de forma a atuar independente do
	dispositivo.
	Empresas de monitoramento de frotas poderiam utilizar o sistema proposto para
	garantir a qualidade do comportamento de seus motoristas. Neste cenário, a melhoria
	necessária seria:
	Adicionar permissões de acesso aos usuários administradores da frota, que poderiam
	visualizar o histórico de toda a frota cadastrada.
	Desta maneira, com a implementação das funcionalidades especificadas acima,
	pode-se criar maneiras de criar renda e monetizar o sistema proposto.
	
	
	
	
	O projeto desenvolvido tem o intuito de possibilitar a interação imersiva com um ambiente remoto através do uso de câmeras estereoscópicas e da operação de um braço robótico. Analisando o projeto, encontramos usos distintos para suas partes componentes em separado, assim como usos envolvendo as duas partes.
	
	Para as câmeras, podemos 
	Vigilância de ambientes, 
	ao utilizar mais de um conjunto de câmeras, uma vez que o usuário poderia assinalar um endereço independente para cada um deles.
	
	Jogos de realidade aumentada, 
	ao possibilitar a movimentação baseada na cabeça do usuário, podemos 
	
	acesso a áreas de acesso restrito em exposições/zoológicos/aquários
	Áreas sensíveis a concentração de pessoas/visão de locais que normalmente não permitiriam o acesso ao público
	
	Braço mecânico
	Programação de braços => necessária forma de gravar os dados
	
	Conjunto
	Desarmamento de bombas
	Operação de reconhecimento em locais remotos
	Operações médicas => maior precisão do hardware/software
	
	\section{Visão de informação}\label{sec-info}
	\subsection{Elementos de informação do sistema}\label{subsec-elementos-info}
	\subsection{Manipulação de informações}\label{subsec-manip-info}
	\subsection{Fluxo de informações}\label{subsec-fluxo-info}
	\section{Visão Computacional}\label{sec-comp}
	\subsection{Requisitos}\label{subsec-req}
	\subsubsection{Requisitos funcionais}\label{subsec-req-func}
	\subsubsection{Requisitos não funcionais}\label{subsec-req-nfunc}
	\section{Visão da Engenharia/Infraestrutura}\label{sec-eng}
	\subsection{Arquitetura do sistema}\label{subsec-arq}
	\section{Visão Tecnológica}\label{sec-tec}
	
	