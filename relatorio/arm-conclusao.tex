% ---
% Arquivo com a conclusão do Trabalho de Conclusão de Curso dos alunos
% Daniel Noriaki Kurosawa
% da Escola Politécnica da Universidade de São Paulo
% ---
	\chapter{Conclusão}\label{cap-conclusao}
	
	O sistema desenvolvido permitiu a interação do usuário com um ambiente através da coleta de dados de movimento tanto por tecnologia vestível como por tecnologia sem marcadores e a imersão visual estereoscópica neste ambiente, cumprindo com grande parte das especificações do projeto.\par

	Entretanto, o escopo do trabalho proposto inicialmente foi reduzido para entrega
	deste documento. Assim, os testes que antes seriam realizados com o sistema conectado a internet foram executados somente em rede local. \par
	
	Isto pode ser explicado pelos desafios enfrentados durante o desenvolvimento do
	projeto, tais como:
		\begin{itemize}
	 \item Utilização de tecnologias diferentes: cada componente foi desenvolvido em plataformas e linguagens distintas, o que demandou um considerável tempo de familiarização e estudo. Além disso, houveram mudanças de hardware durante o projeto, o que exigiu que alguns códigos tivessem que ser mudados
	 
	 \item Integração dos componentes do sistema: a comunicação entre os componentes, embora utilize técnicas e protocolos existentes, foi consideravelmente trabalhosa e de difícil depuração.
	
	
	\item Ausência de certos componentes no mercado brasileiro: Parte do hardware teve que ser importada ou construída especialmente para o projeto, tirando o foco temporariamente da programação e causando atrasos generalizados no projeto.
	\end{itemize}
	
	Algumas sugestões de trabalhos futuros para aprimoramento deste projeto são:
	\begin{itemize}
	\item Identificação e seleção do usuário a operar o sistema;
	
	\item Gravação dos dados de movimento do usuário para execução de tarefas repetitivas;
	
	\item  Uso de motores com feedback da posição atual para medida de precisão e implementação de algoritmos de controle;
	
	 \item Armazenamento das imagens obtidas pelas câmeras;
	 \item Adaptação do código a outros tipos de atuadores;
	
	 \item Adição de um sistema de locomoção para o sistema;
	 \item Configuração de um servidor conectado à internet para aumento da distância de operação;
	 \item Adição de conexão GSM, para controle independente de roteadores locais.
	 \item  Melhorias de hardware/software para melhora de precisão;
	 
	\end{itemize}

	Finalmente, foi possivel concluir que existe uma grande possibilidade no desenvolvimento para o método de operação de braço robótico estudado em aplicações futuras, mas que no entanto, no estado atual da tecnologia, usos comerciais ainda se provam inviáveis devido a alta incidência de erros de leitura.Foi possível concluir também, que a estereoscopia pode ter aplicações imediatas, com grandes possibilidades de desenvolvimento para um futuro próximo\par
	
	\section*{Comentários finais}\label{sec-comentarios}
	Devido ao caráter interdisciplinar deste projeto, não só foi possível como necessária a aquisição de muitos conhecimentos, técnicas e termos da engenharia de computação, sendo todo o processo do projeto marcado pelo aprendizado.\par
	
	Tal característica do projeto reflete o ideal engenheiro de busca pelo conhecimento que permeia todo o curso na Escola Politécnica.\par
	
	Finalmente, julgo como satisfatória e gratificante a experiência de poder realizar todas as etapas de um projeto de engenharia, desde sua concepção até sua conclusão e análise dos resultados, podendo observar em primeiro plano a execução e aplicação de conceitos teóricos vistos (e não vistos) em aula. 
	
	