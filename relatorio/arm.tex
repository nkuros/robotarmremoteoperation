% ------------------------------------------------------------------------
% ------------------------------------------------------------------------
% Teleoperação de braço robótico por rastreamento de movimentos e imersão visual - Trabalho de Conclusão de Curso da Escola Politécnica da USP
% Integrante: Daniel Noriaki Kurosawa
% ------------------------------------------------------------------------ 
% ------------------------------------------------------------------------

\documentclass[
	12pt,				% tamanho da fonte
	openright,			% capítulos começam em pág ímpar
	twoside,			% para impressão em recto e verso. Oposto a oneside
	a4paper,			% tamanho do papel. 
	hyphens,			% url longa nas referencias
	english,			% idioma adicional para hifenização
	brazil				% o último idioma é o principal do documento
]{abntex2}

% ---
%  PACOTES
% ---

% Pacotes básicos 
\usepackage{lmodern}			% Usa a fonte Latin Modern			
\usepackage[T1]{fontenc}		% Selecao de codigos de fonte.
\usepackage[utf8]{inputenc}		% Conversão automática dos acentos
\usepackage{lastpage}			% Usado pela Ficha catalográfica
\usepackage{indentfirst}		% Indenta o primeiro parágrafo de cada seção.
\usepackage{color}				% Controle das cores
\usepackage{amssymb}			% Less than or equal to
\usepackage{tikz}				% desenhos de graficos aux
\usepackage{pgfplots}			% desenhos de gráficos
\usepackage{graphicx}			% Inclusão de gráficos
\usepackage{subcaption}			% Subfigure
\usepackage{float}				% Gráficos
\usepackage{gensymb}			% degree
\usepackage{microtype} 			% para melhorias de justificação
\usepackage{lipsum}				% para geração de dummy text
\usepackage{listings}			% para listagem de código %
\usepackage{amsmath}			% para construção de sistemas lineares%


% Pacotes de citações
\usepackage[brazilian,hyperpageref]{backref}	 % Paginas com as citações
\usepackage[alf,abnt-url-package=url]{abntex2cite}	% Citações padrão ABNT

% --- 
% CONFIGURAÇÕES DE PACOTES
% --- 

% ---
% Configurações do pacote 
\newenvironment{spmatrix}[1]{
	\def\mysubscript{#1}\mathop\bgroup\begin{pmatrix}
	}
{\end{pmatrix}\egroup_{\textstyle\mathstrut\mysubscript}}
% ---

% ---
% Configurações do pacote PGFPLOT
\pgfplotsset{compat=1.14}
\usetikzlibrary{patterns}
\usepgfplotslibrary{fillbetween}
\newcommand{\plotsin}[2]{\begin{axis}[
		xmin=0, xmax=25,
		ymin=-2, ymax=2,
		axis lines=middle,
		%xticklabels={,,},
		%ytick={#1},	
		%yticklabels={#2},
		ylabel near ticks,
		ylabel style={yshift=-0.5cm},
		xlabel shift = 2 pt,
		xlabel={[$t$]},
		ylabel={[$V$]}
		]}
% ---
% Configurações do pacote float
\newfloat{chart}{tbhp}{loc} %section
\floatname{chart}{Gráfico}
\floatstyle{plaintop}
\restylefloat{chart}
\newcommand{\listofcharts}{\listof{chart}{Lista de Gráficos}}
% ---
% Configurações do pacote backref
\renewcommand{\backrefpagesname}{Citado na(s) página(s):~}
\renewcommand{\backref}{}
\renewcommand*{\backrefalt}[4]{
	\ifcase #1 %
	Nenhuma citação no texto.%
	\or
	Citado na página #2.%
	\else
	Citado #1 vezes nas páginas #2.%
	\fi}%
% ---

% ---
% Informações de dados para CAPA e FOLHA DE ROSTO
% ---
\titulo{Imersão em ambiente remoto através da operação de braço robótico por rastreamento de movimentos sem marcadores e sistema de câmeras estereoscópicas}
\autor{Daniel Noriaki Kurosawa}
\local{São Paulo}
\data{2016}
\instituicao{Universidade de São Paulo -  USP \par
			Escola Politécnica \par	
			Engenharia Elétrica com ênfase em Automação e Controle}
\tipotrabalho{Trabalho de Conclusão de Curso}
\preambulo{Trabalho de Formatura apresentado ao Departamento de Engenharia de Telecomunicações e Controle da Escola Politécnica da Universidade de São Paulo para obtenção do Diploma de Engenheiro}
% ---

% ---
% Configurações de aparência do PDF final
\definecolor{blue}{RGB}{41,5,195} % cor azul

% informações do PDF
\makeatletter
\hypersetup{
	%pagebackref=true,
	pdftitle={\@title}, 
	pdfauthor={\@author},
	pdfsubject={\imprimirpreambulo},
	pdfcreator={LaTeX with abnTeX2},
	pdfkeywords={abnt}{latex}{abntex}{abntex2}{trabalho acadêmico}, 
	colorlinks=true,       		% false: boxed links; true: colored links
	linkcolor=blue,          	% color of internal links
	citecolor=blue,        		% color of links to bibliography
	filecolor=magenta,      		% color of file links
	urlcolor=blue,
	bookmarksdepth=4
}
\makeatother
% --- 

% --- 
% Espaçamentos entre linhas e parágrafos 
% --- 
\setlength{\parindent}{1.3cm} % O tamanho do parágrafo
\setlength{\parskip}{0.2cm}  % espacamento entre um paragrafo e outro

% ---
% compila o indice
% ---
\makeindex
% ---

% ----
% Início do documento
% ----
\begin{document}
	
	% Seleciona o idioma do documento
	\selectlanguage{brazil}
	
	% Retira espaço extra obsoleto entre as frases.
	\frenchspacing 
	
	% ----------------------------------------------------------
	% ELEMENTOS PRÉ-TEXTUAIS
	% ----------------------------------------------------------
	% \pretextual
	
	% ---
	% Capa
	% ---
	\imprimircapa
	% ---
	
	% ---
	% Folha de rosto
	% ---
	\imprimirfolhaderosto*
	% ---
	
	% ---
	% Ficha Catalográfica
	% Quando receber pdf da biblioteca, descomentar linhas abaixo
	% ---
	\include{ficha-catalografica}
	% ---
	
	% \begin{fichacatalografica}
	%     \includepdf{fig_ficha_catalografica.pdf}
	% \end{fichacatalografica}

	% ---
	% TODO APROVACAO
	% Quando receber pdf com assinaturas, descomentar linha abaixo
	% ---
	% \includepdf{folhadeaprovacao_final.pdf}
	%
	
	% ---
	% Agradecimentos
	% ---
	\begin{agradecimentos}
	Agradeço imensamente:\par
	À instituição de ensino e todos os seus docentes que não só contribuiram para minha formação acadêmica, mas que através de lições e ensinamentos, contribuiram para minha formação como cidadão.\par
	Ao orientador, Prof. Dr. Reginaldo Arakaki e ao co-orientador  Prof. Dr Fuad Kassab Junior, por todo o apoio e paciência. O bom humor e a confiança depositados foram foram grande fonte de motivação durante o decorrer deste trabalho.\par
	Aos engenheiros Marcelo Angelo Pita e Leandro Rodrigues de Souza, por toda
	orientação e o auxílio prestado na implementação do projeto.\par
	À Camila Aya Uratsuka, por contribuir com as figuras que ilustram este trabalho.\par
	À André Hashimoto Oku, por fornecer hardware usado durante a fase de testes deste projeto.\par

	À minha família pelo amor, incentivo e apoio incondicional.\par
	Por fim aos amigos, pelos constantes incentivos.	\par
	

	
	\end{agradecimentos}
	% ---
	
	% ---
	% Epígrafe
	% ---
	\begin{epigrafe}
		\vspace*{\fill}
		\begin{flushright}
			\textit{``The only difference between screwing around and science ,\\
				...is writing it down."\\
				(Adam Savage, Mythbusters)}
		\end{flushright}
	\end{epigrafe}
	% ---
	
	% ---
	% RESUMOS
	% ---
	% ---
% Arquivo com os resumos do Trabalho de Conclusão de Curso do aluno
% Daniel Noriaki Kurosawa
% da Escola Politécnica da Universidade de São Paulo
% ---

% resumo em português
\setlength{\absparsep}{18pt} % ajusta o espaçamento dos parágrafos do resumo
\begin{resumo}
	Este trabalho estuda a interação imersiva em um ambiente remoto através do uso de estereoscopia e de operação ótica sem marcadores de um braço robótico.
	
	Para tal objetivo, foi definido um projeto cuja execução gerou um sistema composto de:
	\begin{itemize}
		\item Microsoft Kinect, que envia os dados capturados a um braço robótico através de um servidor.
		\item Um sistema de câmeras estereoscópicas que envia suas imagens a um servidor, que podem ser acessadas remotamente por um óculos de realidade virtual ou qualquer dispositivo que tenha um browser compatível instalado.
		\item Um sistema de captura de movimentos da cabeça do usuário via giroscópio, que envia estes dados remotamente a um conjunto de motores localizados na base da câmera, de modo a sincronizar os movimentos da cabeça do usuário a visão das câmeras.		
	\end{itemize}

	 Os resultados finais indicam um grande espaço para o desenvolvimento de projetos deste tipo em aplicações futuras, apesar da necessidade de melhora de precisão para usos comerciais.
	
	\textbf{Palavras-chave}: Internet of Things. Microsoft Kinect. Estereoscopia. Braço Robótico. IoT. 
\end{resumo}

% resumo em inglês
\begin{resumo}[Abstract]
	\begin{otherlanguage*}{english}
		This work studies the immersive interaction with a remote enviroment by using stereoscopy and markerless operation of a robotic arm.


Para tal objetivo, foi definido um projeto cuja execução gerou um sistema composto de:
which create a system that comprises:
\begin{itemize}
	\item A Microsoft Kinect, which sends captured data to a robot controller via server.
	\item A stereoscopic camera system, which sends its images to a server, which are in turn accessed by VR Glasses or any device with a compatible browser installed.
	\item A gyroscope based head-tracking system, which sends user's head pan/tilt data to motors located in the camera system's base, allowing the cameras to move in synchronization with the user's head movements.		
\end{itemize}
Final results indicate great applicability for this approach in future uses. In it's current state however, this approach lacks precision necessary for commercial use.
	
		\vspace{\onelineskip}
		
		\noindent 
		\textbf{Keywords}: Internet of Things. Microsoft Kinect. Stereoscopy.  Robotic Arm. IoT. 
	\end{otherlanguage*}
\end{resumo}
	% ---
	
	% ---
	% inserir lista de ilustrações
	% ---
	\pdfbookmark[0]{\listfigurename}{lof}
	\listoffigures*
	\cleardoublepage
	% ---
	
	% ---
	% inserir lista de tabelas
	% ---
	\pdfbookmark[0]{\listtablename}{lot}
	\listoftables*
	\cleardoublepage
	% ---
	
	% ---
	% inserir lista de gráficos
	% ---
	\pdfbookmark[0]{\listtablename}{lot}
	\listofcharts
	\cleardoublepage
	% ---
	
	% ---
	% inserir lista de abreviaturas e siglas
	% ---
	\begin{siglas}
		\item[CSA] Canadian Space Agency
		\item[ANFIS] Adaptive Neuro-Fuzzy Inference System 
		\item[PID] Controlador Proporcional-Integrativo-Derivativo
		\item[IoT] Internet of Things
		\item[IP] Internet Protocol	
		\item[RM-ODP] Reference Model of Open Distributed Processing 
		\item[MPU] Motion Processing Unit
		\item[I2C] Inter-Integrated Circuit
		\item[PIV]  Controlador proporcional-integrativo-velocidade
		\item[WiFi] Wireless Fidelity
		\item[USB] Universal Serial Bus
		\item[HTML] Hypertext Transfer Protocol 
		\item[TCP-IP] Transmission Control Protocol - Internet Protocol
		\item[UDP] User Datagram Protocol
		\item[IDE] Integrated Development Environment
		\item[PWM] Pulse Width Modulation
		\item[JPEG] Joint Photographic Experts Group
		\item[CI]  Circuito Integrado
		\item[MJPG] Motion JPEG
		\item[RAM] Random Access Memory
		\item[CPU] Central Processing Unit 
		\item[ADC] Analog Digital Converter
		\item[PC] Personal Computer 
		\item[FIFO]  First In, First Out
		\item[J2ME] Java 2 Plataform, Micro Edition
		\item[J2SE] Java 2 Platform, Standard Edition
		\item[J2EE]  Java 2 Platform, Enterprise Edition
		\item[JVM] Java Virtual Machine
		\item[SMC] Simple Managed C
		\item[JPL] Jet Propulsion Laboratory
		\item[NASA] National Aerospace Agency
		\item[VR] Virtual Reality
	\end{siglas}
	% ---
	
	% ---
	% inserir lista de símbolos
	% ---
	\begin{comment}
	\begin{simbolos}
		\item[$ \Gamma $] Letra grega Gama
		\item[$ \Lambda $] Lambda
		\item[$ \zeta $] Letra grega minúscula zeta
		\item[$ \in $] Pertence
	\end{simbolos}
	\end{comment}
	% ---
	
	% ---
	% inserir o sumario
	% ---
	\pdfbookmark[0]{\contentsname}{toc}
	\tableofcontents*
	\cleardoublepage
	% ---
	
	% ----------------------------------------------------------
	% ELEMENTOS TEXTUAIS
	% ----------------------------------------------------------
	\textual
	
	% ----------------------------------------------------------
	% Introdução
	% ----------------------------------------------------------
	% ---
% Arquivo com a Introdução do Trabalho de Conclusão de Curso do aluno
% Daniel Noriaki Kurosawa
% da Escola Politécnica da Universidade de São Paulo
% ---

	\chapter{Introdução} 
	\addcontentsline{toc}{chapter}{Introdução}
	Hoje, através do uso de tecnologias de telecomunicação, é possível a transmissão de grandes volumes de informações a ambientes remotos sem a necessidade da presença física das partes envolvidas, tornando a interação prática, rápida e fisicamente segura.\par
	
	 Existem casos porém\cite{uses-arm}\cite{uses-robot}, em que a interação física a distância com um ambiente remoto é necessária devido a periculosidade ou a dificuldades de acesso ao mesmo, que trariam riscos a um ser humano. Para tais tipo de tarefa, cada vez mais há a substituição de seres humanos por máquinas específicas tais como braços robóticos \par
	
	%Braços robóticos possuem uma ampla gama de aplicações,substituindo seres humanos em tarefas em locais de difícil acesso, repetitivas, ou de alta periculosidade, indo seu uso desde o desarmamento de bombas ao uso em missões espaciais\cite{uses-arm}, passando por processos industriais\cite{uses-robot}.\par
	
	 Porém, apesar do aumento na segurança para o usuário, muitas das abordagens para a operação de um braço robótico continuam complexas e anti-intuitivas  necessitando do uso de artifícios como controles remotos\cite{datasheet-caliber} e joysticks\cite{joystick}. \par
	 
	 Uma segunda linha de pensamento tenta se aproveitar de semelhanças físicas entre o corpo humano e o braço robótico e faz uso, por exemplo, de sensores \cite{wearable} e marcadores\cite{tracker-based} presos ao corpo. Entretanto, essas abordagens podem se provar incomodas ou restritivas em relação à movimentação do usuário devido por exemplo, a ocultação de um marcador ou a rigidez de um sensor vestível.\cite{kinect-based}\par 
	 
	 Uma solução para essas restrições, estudada neste trabalho é o uso de tecnologia de rastreamento visual sem marcadores, que permite que o usuário não mais se preocupe com as restrições de movimento impostas, tornando o controle mais natural.\cite{kinect-based} \par 
	
	 Muitas vezes também, para a operação remota de tais equipamentos, é necessária a transmissão de imagens do ambiente remoto.\par
	 Este trabalho estuda o uso de um sistema de imersão visual através do uso de um óculos realidade virtual por rastreamento dos movimentos da cabeça do usuário e um sistema de câmeras estereoscópicas ao projeto, com a finalidade de criar uma experiência imersiva ao usuário.\par 
	
	 
	
	
	
	
	\begin{figure}[ht!]
		\caption{\label{fig_diagrama}Diagrama do projeto}
		\begin{center}
		\includegraphics[width=\textwidth]{projectchart.pdf}	
		\end{center}
		\legend{Fonte: Autor}
	\end{figure}
	

	
	\section{Motivação}\label{sec-motivacao}
	

	
	Este projeto surge da ideia de tentar unir conceitos de engenharia de automação, ênfase da graduação do autor, a conceitos engenharia de computação, área de pesquisa do professor orientador do projeto.
	
	Dentro deste contexto multidisciplinar, naturalmente a ideia de um projeto de Internet of Things surge, sendo então discutidas ideias sobre possíveis tema para estudo. Entre as alternativas estudadas, experiências do Jet Propulsion Laboratory da NASA\cite{nasa-project} surgem como motivação para este trabalho.
	 

		\section{Ideias iniciais e alternativas do projeto}\label{subsec-iniciais}
	
	Inicialmente, a pesquisa inicial tratou os seguintes temas:\par
	
	
	\begin{itemize}[noitemsep]
		\item Geração de trajetória para braço robótico usando Microsoft Kinect;
		\item Captura de movimentos usando giroscópio;
		\item Algoritmos anti-drifting de giroscópio;
		\item Captura de imagens estereoscópicas em tempo real;
		\item Arduino;
		\item Raspberry Pi;
		\item Particle Photon;
		\item Internet of Things;
	\end{itemize}
	As ideias iniciais continham premissas que envolviam o uso de redes neurais para o controle do braço mecânico, utilizando um controlador ANFIS-PID para o controle dos motores, chegando a ser testado.\par
	Posteriormente, em virtude de limitações de memória do hardware e a complexidade adicionada ao projeto, os algoritmos de redes neurais foram excluídos do escopo do projeto.
	O formato final é tratado na seção seguinte(\autoref{sec-objetivo}).
	\section{Objetivo}\label{sec-objetivo}

	Este estudo propõe um  sistema de imersão e interação do usuário em um ambiente remoto através da  operação de um braço mecânico e do uso de um sistema de imersão visual.\par

	A face de interação do projeto é representada pela operação de um braço robótico à distância, realizada através do uso do sensor de captura de movimento Microsoft Kinect e de um braço robótico conectado a uma rede WiFi.\par
	
	O segundo objetivo do projeto trata da imersão visual do usuário no ambiente remoto
	em que ocorre a interação. Tal imersão será realizada através de um sistema de câmeras estereoscópicas com motores de rotação pan/tilt, uso de óculos de realidade virtual para a recepção das imagens das câmeras e rastreamento de movimentos da cabeça do usuário para o controle de movimentação dos motores.\par
		
	Visando uma maior adaptabilidade de suas partes a outras aplicações, o projeto será encarado de forma modular, de forma que as partes componentes descritas acima poderão ser utilizadas independentemente.\par
	
	 Além disso, o rigor metodológico é uma das preocupações centrais deste trabalho, o que inclui um refinamento na maneira de definir os requisitos do sistema, representado pelo RM-ODP.
	
	% ----------------------------------------------------------
	% ----------------------------------------------------------
	% ----------------------------------------------------------
	% Introdução
	% ----------------------------------------------------------
	% ---
% Arquivo com as especificações do Trabalho de Conclusão de Curso do aluno
% Daniel Noriaki Kurosawa
% da Escola Politécnica da Universidade de São Paulo
% ---
	% ---
		\chapter{Especificações}\label{cap-especificacao}
		
		
	Esta seção baseia-se nos 5 pontos fundamentais da ODP (Open Distributed Processing)	para especificar o projeto a ser desenvolvido. Assim, divide-se este capítulo em:	Visão Empresarial, de Informação, Computacional, de Engenharia/Infraestrutura e de Tecnologia.\par
		\section{Visão Empresarial}\label{sec-empresarial}
	\subsection{Monetização do projeto}\label{subsec-monetizacao}
		O projeto desenvolvido tem o intuito de possibilitar a interação imersiva com um ambiente remoto através do uso de câmeras estereoscópicas e da operação de um braço robótico. Analisando o projeto, encontramos usos distintos para suas partes componentes em separado, assim como usos envolvendo as duas partes.
		
		\subsubsection{Sistema de câmeras estereoscópico}	\label{subsubsec-cameras}
			\begin{itemize}
	\item Câmeras de Vigilância de Ambientes: é possível aplica-los ao utilizar mais de um conjunto de câmeras, uma vez que o usuário tem a possibilidade de assinalar um endereço independente para cada um deles.
	
	\item Câmeras de Entretenimento: ao possibilitar o acompanhamento visual baseado na movimentação da cabeça do usuário, é possível dar acesso a áreas restritas de locais como exposições, zoológicos, aquários e áreas sensíveis a concentração de pessoas através das câmeras instaladas, permitindo uma experiência exclusiva em tais locais. 
	\end{itemize}
		\subsubsection{Braço mecânico}\label{subsubsec-braco}
	\begin{itemize}
	\item Programação de braços robóticos industriais : Caso seja desenvolvido um software para o armazenamento dos dados de movimentação do usuário, é possível a utilização do projeto para uma rápida automatização de uma tarefa repetitiva.
	\end{itemize}	
	\subsubsection{Projeto como um todo}\label{subsubsec-all}
	\begin{itemize}
	\item Desarmamento de Explosivos: pode ser utilizado em operações de reconhecimento em locais remotos e no desarmamento de explosivos, proporcionando uma maior segurança aos operadores.

    \item Operações Médicas à Distância:  Este tipo de abordagem pode oferecer novas alternativas para operações à distância mais eficazes e flexíveis, dando a locais com escassez de recursos acesso a serviços médicos. Para tal aplicação no entanto, são necessárias melhorias na precisão do hardware/software.
    
	\item Jogos de Realidade Aumentada: o projeto pode ser viabilizado neste campo, dando uma experiência imersiva ainda mais enriquecedor ao usuário não só no ambiente virtual, mas também no ambiente real mostrando um grande potencial na criação de novas dinâmicas de jogos e entretenimento.
	\end{itemize}
	
	

	


	
	\section{Visão de Informação}\label{sec-info}
	\subsection{Elementos de informação do sistema}\label{subsec-elementos-info}
	
		\begin{figure}[h!]
		\caption{\label{fig_eleminfo}  Elementos de informação do sistema}
		\begin{center}
			\includegraphics[width=\textwidth]{eleminfo.pdf}	
		\end{center}
		\legend{Fonte: Autor}
	\end{figure}
	
	Para facilidade de compreensão, esta subseção se divide em três partes:
	\subsubsection{Rotação e aceleração da cabeça do usúario}\label{subsubsec-elementos-info-rotation}
	Os movimentos verticais e horizontais da cabeça do usuário são coletados e convertidos em valores numéricos de aceleração e ângulo através do giroscópio GY-521 acoplado na parte traseira do óculos de realidade virtual(\autoref{fig_eleminfo}).
	 
	\subsubsection{Imagens do ambiente remoto}\label{subsubsec-elementos-info-images}	
	As imagens são coletadas por duas câmeras e enviadas ao servidor(Raspberry Pi)(\autoref{fig_eleminfo}). 	

	\subsubsection{Ângulos do braço do usuário}\label{subsubsec-elementos-info-angles}	
	 Os movimentos são coletadas através de um sensor de captura de movimentos e enviadas a um computador, que serve como um segundo servidor(\autoref{fig_eleminfo}).
	\subsection{Manipulação de informações}\label{subsec-manip-info}

	\subsubsection{Braço robótico}\label{subsubsec-elementos-manip-arm}		
	As informações de posição espacial do usuário são enviadas pelo sensor de movimentos a um computador, que por sua vez converte os dados em ângulos e retransmite ao controlador, responsável por fazer a ponte entre o software e os motores do braço mecânico(\autoref{fig_maniparm}).
		\begin{figure}[h!]
		\caption{\label{fig_maniparm}  Manipulação de dados pelo sistema do braço robótico}
		\begin{center}
			\includegraphics[width=\textwidth]{maniparm.pdf}	
		\end{center}
		\legend{Fonte: Autor}
	\end{figure}
	
	\subsubsection{Câmeras}\label{subsubsec-elementos-manip-cam}
	As imagens do ambiente remoto são capturadas por duas câmeras paralelas e enviadas a um servidor(\autoref{fig_manipcam}).
		\begin{figure}[h!]
		\caption{\label{fig_manipcam}  Manipulação de dados pelo sistema de câmeras estereoscópicas}
		\begin{center}
			\includegraphics[width=\textwidth]{manipcam.pdf}	
		\end{center}
		\legend{Fonte: Autor}
	\end{figure}
	
	\subsubsection{Movimentação das Câmeras}\label{subsubsec-elementos-manip-cam-motors}
	Os dados de movimentação da cabeça do usuário são enviados para um Transmissor de ângulo de rotação(\autoref{fig_manipmotcam}).
			\begin{figure}[h!]
		\caption{\label{fig_manipmotcam}  Manipulação de dados pelo sistema de motores da câmera}
		\begin{center}
			\includegraphics[width=\textwidth]{manipmotcam.pdf}	
		\end{center}
		\legend{Fonte: Autor}
	\end{figure}
	
	\subsection{Fluxo de informações}\label{subsec-fluxo-info}
	No projeto, todo o fluxo de informações entre os nós da arquitetura é unidirecional. Na \autoref{fig_techcom}, em cada flecha de comunicação, é mostrada de forma genérica a tecnologia utilizada entre os nós para troca de informações.
	As tecnologias para intercomunicação são as seguintes:
		\begin{figure}[h!]
		\caption{\label{fig_techcom}  Tecnologias Utilizadas para Conexão entre os Nós}
		\begin{center}
			\includegraphics[width=\textwidth]{techcom.pdf}	
		\end{center}
		\legend{Fonte: Autor}
	\end{figure}
	
	\begin{itemize}[noitemsep]
		\item PWM - Link Cabeado: Comunicação entre os motores e o receptor de ângulos da câmera.
		\item PWM - Comunicação entre os motores e o receptor de ângulos do braço.
		\item Serial USB - Link Cabeado: Comunicação entre as câmeras e o servidor.
		\item Serial USB - Link Cabeado: Comunicação entre sensor de captura de movimento e o computador(servidor)
		\item Serial I2C - Link Cabeado: Comunicação entre giroscópio GY521 e Particle Photon. 
		\item HTTP TCP-IP- Link Internet: Uso do protocolo HTTP e TCP-IP para comunicação entre dispositivo móvel e servidor.
		\item Wireless UDP- Link Wireless: Uso do protocolo UDP  para comunicação entre transmissor de dados de rotação da cabeça do usuário e o controlador dos motores da câmera.
		\item Wireless UDP- Link Wireless: Uso do protocolo UDP  para comunicação entre o computador e o controlador do braço robótico.

	\end{itemize}

	As mensagens trocadas entre cada nó são detalhadas a seguir:
	
	\begin{itemize}
		\item Transmissor e Receptor da câmera: O transmissor de dados de rotação da cabeça do usuário envia ao o controlador dos motores da câmera mensagens com o seguinte formato:\par
		ângulos : { “rotação horizontal” : dados[0], “rotação vertical” : dados[1]}
\par
		\item  Giroscópio e Transmissor de ângulos de rotação: O giroscópio envia ao transmissor de dados de rotação da cabeça do usuário quaternion com o seguinte formato:
				ângulos : { “yaw” : dados[0], “pitch” : dados[1],“row” : dados[2], “rotation” : dados[3]}
		
		\item Transmissor e Receptor da câmera: O transmissor de dados de rotação da cabeça do usuário envia ao o controlador dos motores da câmera mensagens com o seguinte formato:\par
		ângulos : { “motor da base” : dados[0], “motor do ombro” : dados[1],“motor do cotovelo” : dados[2], “motor do pulso” : dados[3],“motor de rotação do pulso” : dados[4], “motor da garra” : dados[5]}
		
		\item  Receptor(ambos) e Motores: O receptor converte os dados recebidos em comandos PWM para os servo motores.
		
		\item  Câmeras e Servidor: As câmeras enviam imagens no formato JPEG para o computador em que o servidor se encontra(Raspberry Pi), que são coletadas pelo servidor.
		
		\item  Smartphone e Servidor: A aplicação utiliza o protocolo HTTP para realizar requisições ao servidor Apache pela internet.
		

		
\par

	\end{itemize}


	\section{Visão Computacional}\label{sec-comp}
	\subsection{Requisitos}\label{subsec-req}
	\subsubsection{Requisitos funcionais}\label{subsec-req-func}
	Nesta seção, apresentam-se os requisitos funcionais e não funcionais do sistema proposto.
\par
	\begin{center}
		\begin{tabular}{ | l |  p{7cm} |}
			\hline
			Identificação & RFB01  \\ \hline
			Nome & Captura de movimentos (braço)\\ \hline
			Descrição & O sistema deve suportar a captura da posição espacial das juntas do corpo do usuário
			\\ \hline
			Explicação & Os dados serão usados para o cálculo dos ângulos de movimentação do braço robótico  
			\\ \hline

		\end{tabular}
	\end{center}

	\begin{center}
	\begin{tabular}{ | l |  p{7cm} |}
		\hline
		Identificação & RFB02  \\ \hline
		Nome & Cálculo de ângulos (braço)\\ \hline
		Descrição & O sistema deve receber a posição espacial das juntas do corpo do usuário e através de trigonometria, calcular os ângulos de cada junta do braço do usuário
		\\ \hline
		Explicação & Os ângulos calculados serão transmitidos para o controlador dos motores do braço robótico  
		\\ \hline
		
	\end{tabular}
\end{center}
		\begin{center}
	\begin{tabular}{ | l |  p{7cm} |}
		\hline
		Identificação & RFB03  \\ \hline
		Nome & Visualização de ângulos\\ \hline
		Descrição & O sistema deve exibir o "esqueleto" do usuário e mostrar os ângulos em cada junta do braço
		\\ \hline
		Explicação & O sistema deve possuir uma interface gráfica que permita a verificação de captura dos dados e seu correto processamento em tempo real.
		\\ \hline
	\end{tabular}
\end{center}
	\begin{center}
	\begin{tabular}{ | l |  p{7cm} |}
		\hline
		Identificação & RFC01  \\ \hline
		Nome & Recepção de imagens(câmera)\\ \hline
		Descrição & O sistema deve receber as imagens das duas câmeras do sistema e disponibilizá-las ao servidor
		\\ \hline
		Explicação & As imagens serão juntadas pelo servidor em uma imagem estereoscópica, que será transmitida ao usuário
		\\ \hline
		
	\end{tabular}
\end{center}
		\begin{center}
		\begin{tabular}{ | l |  p{7cm} |}
			\hline
			Identificação & RFC01  \\ \hline
			Nome & Conversão de imagem\\ \hline
			Descrição & O sistema deve receber a imagem das câmeras e formatá-las de modo apropriado à visão estereoscópica
			\\ \hline
			Explicação & As duas imagens serão disponibilizadas lado a lado, de forma que sejam observadas como uma única imagem com profundidade quando o usuário vestir os óculos de realidade virtual.
			\\ \hline
		\end{tabular}
		\end{center}
				\begin{center}
				\begin{tabular}{ | l |  p{7cm} |}
					\hline
					Identificação & RFM01  \\ \hline
					Nome & Cálculo de ângulos(motores da câmera)\\ \hline
					Descrição & O sistema deve receber os dados de rotação e aceleração do giroscópio e calcular os ângulos em cada eixo com as devidas correções
					\\ \hline
					Explicação & Os ângulos calculados serão transmitidos para o controlador dos motores do braço robótico  
					\\ \hline
					
				\end{tabular}
			\end{center}


	\begin{center}
	\begin{tabular}{ | l |  p{7cm} |}
		\hline
		Identificação & RFM02  \\ \hline
		Nome & Cálculo de ângulos (motores da câmera)\\ \hline
		Descrição & O sistema deve receber a posição espacial das juntas do corpo do usuário e através de trigonometria, calcular os ângulos de cada junta do braço do usuário
		\\ \hline
		Explicação & Os ângulos calculados serão transmitidos para o controlador dos motores do braço robótico  
		\\ \hline
		
	\end{tabular}
\end{center}

	\subsubsection{Requisitos não funcionais}\label{subsec-req-nfunc}
		\begin{center}
	\begin{tabular}{ | l |  p{7cm} |}
		\hline
		Identificação & RNFB01  \\ \hline
		Nome & Taxa de envio de dados (braço)\\ \hline
		Descrição & O hardware de comunicação deve otimizar o número de requisições enviadas ao hardware de controle do braço.
		\\ \hline
		Explicação &  A taxa de envio de dados deve ser suficiente para que o o intervalo entre envios não seja perceptível, sem que sobrecarregue o hardware de controle.
		\\ \hline
		
	\end{tabular}
\end{center}	
		\begin{center}
	\begin{tabular}{ | l |  p{7cm} |}
		\hline
		Identificação & RNFB01  \\ \hline
		Nome & Taxa de envio de dados (Câmeras)\\ \hline
		Descrição & O hardware de captura de imagens deve otimizar o número de imagens enviadas ao servidor.
		\\ \hline
		Explicação &  A taxa de envio de dados deve ser suficiente para que o intervalo entre frames não seja perceptível, sem que sobrecarregue o servidor.
		\\ \hline
		
	\end{tabular}
\end{center}
	
		\begin{center}
		\begin{tabular}{ | l |  p{7cm} |}
			\hline
			Identificação & RNFM01  \\ \hline
			Nome & Taxa de envio de dados (motores da câmera)\\ \hline
			Descrição & O hardware de comunicação deve otimizar o número de requisições enviadas ao hardware de controle dos motores da câmera.
			\\ \hline
			Explicação &  A taxa de envio de dados deve ser  suficiente para que o intervalo entre envios não seja perceptível, sem que sobrecarregue o hardware de controle.
			\\ \hline
			
		\end{tabular}
	\end{center}

	\section{Visão da Engenharia/Infraestrutura}\label{sec-eng}
	Esta seção apresenta a arquitetura definida para a implementação do sistema.
	\subsection{Arquitetura do sistema}\label{subsec-arq}	
	\subsubsection{Sistema de Operação do Braço Mecânico}\label{subsec-arq-arm}

	\begin{figure}[h!]
		\caption{\label{fig_arq-arm}  Arquitetura do sistema de operação do braço robótico}
		\begin{center}
			\includegraphics[width=\textwidth]{arq-arm.pdf}	
		\end{center}
		\legend{Fonte: Autor}
	\end{figure}
	
	\paragraph{Hardware de captura de dados}\label{par-arq-armcapture}
	É o hardware que realiza a interface com o usuário.
	Trata-se da parte responsável por coletar os dados relevantes de movimentação do braço do usuário, convertê-los para ângulos e enviá-los ao controlador do motor via protocolo UDP.\par

		\paragraph{Hardware de controle dos motores}\label{par-arq-armtranslate}
	É o hardware que realiza a interface com os motores, convertendo os dados recebidos em comandos PWM usados pelos motores.\par

	\subsubsection{Sistema de Operação do sistema de motores da câmera}\label{subsec-arq-mot}	

		\begin{figure}[h!]
		\caption{\label{fig_arq-mot}   Arquitetura do sistema de motores da câmera}
		\begin{center}
			\includegraphics[width=\textwidth]{arq-mot.pdf}	
		\end{center}
		\legend{Fonte: Autor}
	\end{figure}
	\paragraph{Hardware de captura de dados}\label{par-arq-motcapture}
	Analogamente ao hardware do braço, é o hardware que realiza a interface com o usuário.\par
	Trata-se da parte responsável por coletar os dados relevantes de movimentação do braço do usuário, convertê-los para ângulos e enviá-los ao controlador do motor via protocolo UDP.\par
	\paragraph{Hardware de controle dos motores}\label{par-arq-mottranslate}
	Novamente, é o hardware que realiza a interface com os motores, convertendo os dados recebidos em comandos PWM usados pelos motores.\par

	\subsubsection{Sistema de Imagens Estereoscópicas}\label{subsec-arq-cam}
		\begin{figure}[h!]
		\caption{\label{fig_arq-cam}   Arquitetura do sistema de captura de imagens estereoscópicas}
		\begin{center}
			\includegraphics[width=\textwidth]{arq-cam.pdf}	
		\end{center}
		\legend{Fonte: Autor}
	\end{figure}
	\paragraph{Hardware de captura de imagens}\label{par-arq-cam}
	Realiza a captura de imagens do ambiente remoto e as disponibiliza para o servidor. 
	\paragraph{Servidor}\label{par-arq-server}
	Recebe as imagens e as reformata para que sejam vistas em um óculos de realidade virtual genérico como uma imagem estereoscópica.
	\paragraph{Hardware de interface gráfica}\label{par-arq-cell}
	Realiza a requisição de imagens ao servidor. Além disso, é responsável pela interface com o usuário, isto é, permite a visualização do ambiente remoto.

	\section{Visão Tecnológica}\label{sec-tec}
		\begin{figure}[h!]
		\caption{\label{fig_tech}  Visão tecnológica de Hardware do sistema}
		\begin{center}
			\includegraphics[width=100mm]{visaotecnologica.pdf}	
		\end{center}
		\legend{Fonte: Autor}
	\end{figure}
	Neste capítulo, são apresentadas as tecnologias utilizadas para cada camada do
	projeto, a partir da arquitetura definida no seção anterior . Posteriormente, são fornecidas informações básicas sobre cada tecnologia. As principais tecnologias utilizadas foram: Microsoft Kinect, Raspberry Pi, Particle Photon, GY521, Visual C\#, Apache HTTP Server, mjpg-streamer.\par
	

	\subsection{Microsoft Kinect}\label{subsec-kinect}
		 \begin{figure}[h!]
		\caption{\label{fig_kinectlogo}  Logotipo Kinect}
		\begin{center}
			\includegraphics[width=100mm]{kinect_logo.png}	
		\end{center}
		\legend{Fonte: Microsoft}
	\end{figure}
	 \begin{figure}[h!]
	\caption{\label{fig_kinect}  Kinect 2.0}
	\begin{center}
		\includegraphics[width=100mm]{kinect20.jpg}	
	\end{center}
	\legend{Fonte: Microsoft}
\end{figure}
	Originalmente chamado Projeto Natal durante seu desenvolvimento, Kinect é uma linha de sensores de captura de movimento desenvolvida pela Microsoft em parceria com a Invensense para seus consoles Xbox 360 e Xbox one.\par

	Baseado em um periférico add-on estilo webcam, ele permite que os usuários controlem e interajam com seus consoles/computadores sem a necessidade de um controle remoto, através de uma interface natural do usuário usando gestos e comando por voz.
	A primeira geração do Kinect foi introduzida em novembro de 2010, com uma tentativa de expandir o público de Xbox 360 além do seu público gamer usual. 
	
	Uma versão para Windows foi lançada em 1 de fevereiro de 2012, e em 16 de junho de 2011 a Microsoft lançou o kit  de desenvolvimento de software do Kinect para Windows 7. Este SDK permitiu a desenvolvedores escreverem apps Kinect em C++/CLI, C\#, ou Visual Basic .NET.
	Atualmente, o Kinect encontra-se em sua segunda versão, com sua SDK na versão 2.0\cite{microsoft-kinect}.

	
	
	\subsection{Particle Photon}\label{subsec-photon}

		 \begin{figure}[h!]
		\caption{\label{fig_particle}  Logotipo Particle}
		\begin{center}
			\includegraphics[width=100mm]{particle_logo.png}	
		\end{center}
		\legend{Fonte: Particle}
		\end{figure}
	 A Particle surgiu como uma campanha no Kickstarter em 2013 com a visão de desenvolvimento de Internet of Things simples e acessível e atualmente suas ferramentas são utilizados por 70 mil engenheiros em mais de 170 países e por várias companhias Fortune 500 para desenvolver e gerenciar diversos novos produtos IoT. Seu portfólio  de produtos inclui a Particle Cloud, uma infraestura de nuvem própria para os dispositivos da empresa, um cartão SIM e plano de datas próprios, os microcontroladores conectados via nuvem Electron e Photon e software exclusivo para desenvolvimento. A Particle foi listada em 2015 como uma das Fast Company’s Most Innovative Companies  e é listada em vários relatórios da Gartner em soluções IoT. \par
	\begin{figure}[h!]
	\caption{\label{fig_photon}  Particle Photon}
	\begin{center}
		\includegraphics[width=70mm]{photon.jpg}	
	\end{center}
	\legend{Fonte: Particle}
\end{figure}		 	

	A Particle Photon, utilizada neste projeto é uma placa de prototipagem de hardware e software voltada a IoT baseada em um microcontrolador STM32 ARM Cortex M3 e um CI WiFi Broadcom BCM43362 totalmente desenvolvida pela Particle. A placa foi escolhida pela fácil programação, fator de forma e peso reduzidos e pela conectividade WiFi integrada\cite{Particle}. \par

		
	
	
	\subsection{GY521}\label{subsec-GY521}

	

	A placa GY521 é uma placa destinada a tornar a prototipação usando o CI MPU6050 mais fácil e prática ao integrá-lo em uma placa com saídas through hole.
	
	O CI MPU6050 é um dispositivo de rastreamento de movimento que contém um giroscópio de 3 eixos, um acelerômetro de 3 eixos e um Processador digital de movimento(DMP) em um encapsulamento de 4x4x0.9mm.
	 
	A MPU-6050 possui três conversores analógico-digitais de 16 bits (ADCs) para digitalizar as saídas do giroscópio e três 16-bit ADCs para digitalizar as saídas do acelerômetro. Para rastreamento de precisão de movimentos rápidos e lentos,  o giroscópio apresenta ranges programáveis pelo usuário de $\pm$250, $\pm$500, $\pm$1000, e $\pm$2000\degree/sec (dps) e o acelerômetro, ranges de $\pm$2g, $\pm$4g, $\pm$8g, e $\pm$16g. Um buffer on-chip FIFO de 1024 Bytes ajuda na redução de consumo de energia do sistema permitindo ao processador de sistema ler os dados do sensor em bursts e assim, entrar em um modo de baixa potência enquanto o MPU coleta mais dados.
	Comunicação com todos os registros de dispositivos são executadas utilizando ou I2C, ou 400kHz. Recursos adicionais incluem um sensor de temperatura acoplado e um oscilador on-chip com variação $\pm$ 1 \% na faixa de temperatura de operação. A peça possui uma robusta tolerância a choques de 10.000g e possui filtros passa-baixa programáveis para o giroscópio, o acelerômetro e o sensor de temperatura\cite{mpu6050}.
	

	
	\subsection{Raspberry Pi}\label{subsec-rasp}
	
		 \begin{figure}[h!]
		\caption{\label{fig_raspfound}  Logotipo Raspberry Pi}
		\begin{center}
			\includegraphics[width=50mm]{raspberry-pi-logo.png}	
		\end{center}
		\legend{Fonte: Raspberry Pi Foundation}
	\end{figure}
	A ideia para um computador pequeno e barato para crianças surgiu em 2006, quando Eben Upton, Rob Mullins, Jack Lang e Alan Mycroft, do laboratório de computação da Universidade de Cambridge, ficaram preocupados com o declínio do nível dos estudantes do ensino médio que pleiteavam uma vaga para o curso de ciência da computação. Nos anos 90, os estudantes entrevistados para o curso tinham uma vasta experiência como programadores hobbystas, no entanto nos anos 2000 a situação mudou bastante os candidatos em sua maioria tinham algum conhecimento em web design.\par
	
			 \begin{figure}[h!]
		\caption{\label{fig_raspberry}  Raspbery Pi Modelo B versão 3}
		\begin{center}
			\includegraphics[width=100mm]{rasp3.jpg}	
		\end{center}
		\legend{Fonte: Raspberry Pi Foundation}
	\end{figure}
	
	O jeito como as crianças inglesas lidam com a tecnologia mudou. Alguns problemas foram encontrados: um grande número de aulas utilizando Word e Excel, ou escrevendo páginas web; o fim do crescimento da era ponto-com; e o crescimento dos PC’s e consoles de video-game, que substituíram as máquinas que as pessoas das gerações anteriores aprenderem a programar.\par
	Não há muito em que um pequeno grupo possa fazer para solucionar problemas como um currículo inadequado ou o fim de uma bolha financeira. Mas o grupo de Cambridge achou que podia fazer algo para mudar a situação em que os computadores se tornaram custosos e complexos e em que a programação neles teve que ser proibida pelos pais e assim pensaram em uma plataforma, que assim como os computadores pessoais antigos, podiam inicializar em um ambiente de programação. Então de 2006 a 2008, este grupo desenvolveu o que agora se tornou o Raspberry pi.\par
	Em 2008, os processadores desenvolvidos para telefones celulares se tornaram mais
	acessíveis, e com capacidade de processamento suficiente para prover multimídia, um
	recurso que poderia deixar a placa atraente para crianças, que não se interessariam por um dispositivo puramente voltado para programação.\par
	Foi então que foi criada a Fundação Raspberry pi, para transformar o projeto em
	realidade. Três anos depois o Raspberry pi modelo B entrou em produção em massa, e
	em dois anos vendeu mais de dois milhões de unidades.(...)\par
	
	O objetivo do Raspberry é difundir o uso de computadores de baixo custo, que
	possam ser utilizados para programação. Sendo uma tentativa de quebrar o paradigma que
	para ter acesso a internet é necessário comprar um computador de alto custo. Outra meta é a de que o uso dos computadores pessoais seja difundido entre as crianças. \cite{raspberry}.\par


	\subsection{Java}\label{subsec-java}
	
	O desenvolvimento em Processing é realizado utilizando-se Java, linguagem de programação orientada a objetos.
	Figura 31 – Logo da linguagem Java
		 \begin{figure}[h!]
	\caption{\label{fig_java}  Logo da linguagem Java}
	\begin{center}
		\includegraphics[width=100mm]{java-logo-vector.png}	
	\end{center}
	\legend{Fonte: Oracle}
\end{figure}

	\subsubsection{História}\label{subsubsec-histjava}
	Java foi iniciado em junho de 2001, em um projeto liderado por James Gosling, Mike Sheridan e Patrick Naughton, sendo originalmente projetado para televisão interativa.\par
	Inicialmente, a linguagem era chamada Oak, referência ao carvalho (oak) plantado no exterior do escritório de Gosling. Mais tarde, a linguagem foi chamada de Green e, por fim, Java.\par
	A primeira implementação pública do Java (Java 1.0) foi lançada em 1995, pela Sun Microsystems. Nesta época, os principais navegadores incorporaram a execução de Java applets nas páginas web. Posteriormente, com o lançamento da versão Java 2, aumentou o número de plataformas e configurações compatíveis. A versão destinada à Desktops foi denominada JM2SE (Java Standard Edition), enquanto as aplicações de empresas eram desenvolvidas utilizando-se a tecnologia J2EE (Java Enterprise Edition). Por sua vez, a versão J2ME (Java Mobile Edition) oferecia recursos especificamente para aplicações de celulares. Posteriormente, por motivos de marketing, as versões foram renomeadas Java SE, Java EE e Java ME.\par
	Atualmente, o Java SE encontra-se na versão 8.0, e é considerada ainda como uma das mais utilizadas linguagens de programação.\cite{java}

	\subsubsection{Características da linguagem}\label{subsubsec-caractjava}
	O código Java é compilado em um bytecode, que é executado na máquina virtual java (JVM - Java Virtual Machine), independentemente da arquitetura da máquina.
	Possui gestão automática de memória, com uma implementação padrão de “Garbage Collector” que monitora as referências aos objetos em uma aplicação.
	Graças à grande popularidade da linguagem, existem inúmeras bibliotecas e frameworks disponíveis atualmente para o desenvolvimento de aplicações em Java. Dentre
	as funcionalidades oferecidas pelas bibliotecas mais utilizadas, destacam-se: IO (Input/Output), etworking (comunicação em rede), Concurrency (programação paralela),
	segurança, interfaces gráficas, etc.
	
	\subsection{Visual C\#}\label{subsec-csharp}
	 C\# é uma linguagem de programação de propósito geral orientada a objetos criada para o desenvolvimento de uma variedade de aplicações executadas sobre o .NET Framework.\par
	Seu time de desenvolvimento é conduzido por Anders Hejlsberg e sua versão mais recente é C\# 6.0, lançada em 2015.\par
	Já o Visual C\#, uma implementação da linguagem C\# pela Microsoft, é suportado por Visual Studio que possui um editor de código completo, compilador, modelos de projetos, designers, assistentes de código, um depurador poderoso e de fácil manuseio e entre outras ferramentas. \par
	A biblioteca de classes do .NET Framework fornece acesso a vários serviços do sistema operacional.\par
	\subsubsection{História}\label{subsubsec-histcsharp}
	

	Durante o desenvolvimento de .NET Framework, a biblioteca de classes foi originalmente escrita utilizando um sistema compilador de códigos gerenciados chamado Simple Managed C (SMC). Em janeiro de 1999, Anders Hejlsberg forma um time com o objetivo de construir uma nova linguagem que inicialmente foi nomeada Cool e posteriormente alterado para "C-like Object Oriented Language". Apesar de ter sido cogitado manter o nome "Cool" como nome final da linguagem , por questões de marca registrada esta ideia foi abandonada pela  Microsoft.\par
	Em 2000 quando o projeto .NET foi anunciado publicamente, a linguagem já havia sido renomeada para C\# e as bibliotecas de classes e a ASP.NET runtime  já haviam sido transferidoa para esta linguagem. O principal designer e arquiteto da C\# foi Anders Hejlsberg da Microsoft, que já havia participado no desenvolvimento de Turbo Pascal, Embarcadero Delphi (antigos CodeGear Delphi, Inprise Delphi e Borland Delphi), e Visual J++. \par\cite{visualc}
	\subsubsection{Características da linguagem}\label{subsubsec-caractcsharp}

	C\# foi projetado para se adequar a aplicações tanto em sistemas hospedados quanto embarcados, que vão desde grandes aplicações com sistemas operacionais sofisticadas até as menores com funções específicas. 
	Devido a estas características, tanto a linguagem como suas implementações devem fornecer suporte para princípios de engenharia de software como checagem strong type, checagem de array bounds, detecção de tentativas de utilização de variáveis não inicializadas e coletor de lixo automático. Robustez e durabilidade de software e a produtividade do programador são fatores importantes para a linguagem.

	Portabilidade é também uma característica muito importante para códigos-fonte e para programadores, especialmente para aqueles que já estão familiarizados com C and C++. 
		
	Finalmente, embora as aplicações C\# terem sidos planejadas para serem econômicas considerando a memória e requisitos de poder de processamento, a linguagem não se destina a competir diretamente em desempenho e tamanho com C ou Assembly.	
	  
	 
	\subsection{Protocolo HTTP}\label{subsec-http}
	
	HTTP (HyperText Transfer Protocol) é um protocolo da camada de aplicação para comunicação via internet. Trata-se de um protocolo do tipo requisição-resposta, baseado no modelo de computação cliente-servidor, no qual o cliente envia requisições ao servidor que, por sua vez, retorna uma reposta. Por exemplo, um navegador web é um cliente, enquanto um outro computador hospedando uma página web é o servidor.\par
	No contexto do modelo de camadas de redes, HTTP é um protocolo da camada de aplicação e é comumente utilizado em conjunto do TCP (Transmission Control Protocol), protocolo da camada de transporte.
	Dentre os métodos de requisições mais comuns, destacam-se:\par
	GET\par
	Requisita uma representação de um determinado recurso (uma página da internet, por exemplo). Requisições utilizando GET devem apenas recuperar dados e não ter outro efeito.\par
	POST\par
	Requisita que o servidor web aceite e salve os dados enviados no corpo (body) da mensagem enviada. É frequentemente utilizado ao fazer uploads de arquivos ou submeter formulários. Neste projeto, requisições do tipo GET são utilizadas para transmitir as imagens usando o servidor.\par

		\subsection{Servidor Apache HTTP}\label{subsec-apache}
		
					 \begin{figure}[h!]
			\caption{\label{fig_apachelogo}  Logo Apache HTTP Server}
			\begin{center}
				\includegraphics[width=100mm]{Apache_HTTP_server_logo_(2016).png}	
			\end{center}
			\legend{Fonte: Apache Software Foundation}
		\end{figure}
		O  Projeto Servidor Apache HTTP é um esforço de desenvolvimento de software colaborativo destinado a criar implementação de código fonte de servidor HTTP (Web) robusto, comercial, característico e disponível gratuitamente. O projeto é administrado em conjunto por um grupo de voluntários de diversos países, usando a internet e a web para comunicar, planejar e desenvolver o servidor e sua documentação relacionada.
		O Apache HTTP Server é um projeto da Apache Software Foundation.

	\subsubsection{Características}\label{subsubsec-caracteristicasapache}

	Lançado em 1995, o projeto Apache HTTP Server ("httpd") da fundação The Apache Software Foundation tem sido o servidor web mais popular desde abril de 1996, tendo sido celebrado o 20º aniversário do projeto em fevereiro de 2015.\cite{apache}
	
		\subsection{mjpg-streamer}\label{subsec-mjpg-streamer}	
		
		
		
	mjpg-streamer é uma aplicação de linhas de comando criada originalmente por Tom Stöveken para linux que copia frames JPEG de uma ou mais fontes para múltiplos output plugins. Pode ser usada para transmitir arquivos JPEG  através de uma webcam conectada a uma rede baseada em IP para vários tipos de software capazes de receber transmissões MJPG ao mesmo tempo.
	\subsubsection{Características}\label{subsubsec-histmjpg}
	Originalmente escrito para aparelhos embarcados com memórias RAM e CPU limitadas, uvc\_streamer(programa predecessor ao mjpg-streamer)  foi criado devido ao fato de câmeras compatíveis com Linux-UVC produzirem diretamente dados em JPEG, o que  permitia uma transmissão M-JPEG rápida e de com performance considerável.
	 Hoje, mjpg-streamer tem suporte a uma variedade de dispositivos de entrada diferentes.\cite{mjpg}

	% ----------------------------------------------------------
	% ----------------------------------------------------------
	
	% ----------------------------------------------------------
	% Metodologia
	% ----------------------------------------------------------
	
% ---
% Arquivo com a metodologia do Trabalho de Conclusão de Curso do aluno
% Daniel Noriaki Kurosawa
% da Escola Politécnica da Universidade de São Paulo
% ---
	% ---
	\chapter{Metodologia}\label{cap-metodologia}
	% ---
	
	Este capítulo apresenta o método de trabalho adotado para a execução das atividades
	do projeto e divide-se em três partes: organização de tarefas, escolha de hardware e software utilizados.
	
	

	
	% ---
	\section{Planejamento}\label{sec-planejamento}
	% ---
	Inicialmente, foi elaborado um diagrama de Gantt do sistema contendo as principais etapas do projeto (\autoref{fig_gantt}) a partir do qual foram identificadas as tarefas a serem realizadas durante o projeto, que foram posteriormente transferidas para o Trello.
	\begin{figure}[h!]
	\caption{\label{fig_gantt} Diagrama de Gantt do sistema }
		\begin{center}
	\includegraphics[width=\textwidth]{gantt.png}	
\end{center}
	\legend{Fonte: Autor}
\end{figure}

	% ---
	\subsection{Trello}\label{subsec-eap}
	% ---
	Trello é uma ferramenta de organização de projetos  baseada nos quadros Kanban da metodologia ágil. Extremamente versátil, permite uma análise rápida da situação do projeto, das etapas cumpridas e do cronograma geral do projeto. No Trello, é possível criar “boards” que agrupam listas (“lists”) de tarefas que devem ser feitas. Cada tarefa é então representada por um “card”, criado dentro de uma “list”. As “lists” criadas dentro da “board” do projeto são:\par
	
	\subparagraph{Backlog}: Tarefas que ainda precisam ser analisadas e aprovadas para serem colocadas na fase de atividades.
	\subparagraph{Sprint}: Tarefas a serem realizadas após a conclusão das tarefas sendo trabalhadas atualmente.
	\subparagraph{Doing}: Tarefas em execução.
	\subparagraph{To be verified}: Tarefas executadas a serem revisadas.
	\subparagraph{Done}: Tarefas terminadas.
	\subparagraph{Halt}: Tarefas suspensas/canceladas.

		\begin{figure}[h!]
		\caption{\label{fig_trello} Kanban do projeto usando Trello }
		\begin{center}
	\includegraphics[width=\textwidth]{trello.png}	
	\end{center}
	\legend{Fonte: Autor}
	\end{figure}
	
	
	% ---
	\section{Programas e IDEs}\label{sec-requisitos}
	% ---
	Esta seção apresenta os principais programas e IDEs (Integrated Development Environment) utilizados durante o desenvolvimento de cada componente do sistema proposto.
		\subsection{Visual Studio 2015}\label{subsec-VSIDE}
		A aplicação para Kinect foi feita usando-se o programa Visual Studio 2015.\par
		O Visual Studio é a IDE usada pelo Kinect SDK da Microsoft para o desenvolvimento de aplicações Kinect, e permite elaborar o aplicativo e facilmente verificar os seus erros. É também a 
		Possui inúmeros exemplos de código de aplicações em C\#, C++ e Visual Basic que também auxiliam no desenvolvimento de uma aplicação.
		\begin{figure}[h!]
		\caption{\label{fig_VSIDE} Visual Studio 2015 IDE }
		\begin{center}
			\includegraphics[width=\textwidth]{VSIDE.png}	
		\end{center}
		\legend{Fonte: Autor}
		\end{figure}
	
	
		\subsection{Raspberry Pi}\label{subsec-raspIDE}
			A configuração do servidor HTTP Apache foi escrita utilizando-se o editor de texto GNU NANO, que já vem pré-instalado no Raspberry Pi. A escolha se deu pela simplicidade da estrutura do arquivo de configuração, que não demandava grandes funcionalidades.
			\begin{figure}[h!]
		\caption{\label{fig_raspIDE} Editor de texto GNU NANO do Raspberry Pi }
		\begin{center}
			\includegraphics[width=\textwidth]{raspIDE.png}	
		\end{center}
		\legend{Fonte: Autor}
	\end{figure}

	\subsection{Particle IDE}\label{subsec-particleIDE}
	A Programação das placas Photon se deu pelo uso da Particle IDE, ambiente próprio para a programação das mesmas e que pode ser acessado através de um browser, sendo a tradução e envio do código às placas feito via nuvem própria.
			\begin{figure}[h!]
		\caption{\label{fig_particleIDE} Particle IDE  }
		\begin{center}
			\includegraphics[width=\textwidth]{ParticleIDE.png}	
		\end{center}
		\legend{Fonte: Autor}
	\end{figure}
	
	
	% ----------------------------------------------------------
 
	% ----------------------------------------------------------
	% Execução
	% ----------------------------------------------------------
	% ---
% Arquivo com a execução do Trabalho de Conclusão de Curso dos aluno
% Daniel Noriaki Kurosawa 
% da Escola Politécnica da Universidade de São Paulo
% ---
	\chapter{Execução e Resultados obtidos}\label{cap-execucao}
	Ete capítulo apresenta as atividades realizadas resultados atingidos, obtidos a partir da especificação do sistema e da metodologia de trabalho adotada.
	
	\subsection{Controle abre-fecha e rotação da garra}\label{subsec-garra}
	Primeiramente, foi feito o estudo de movimentação garra e de seu controle, uma vez que seu controle é totalmente independente da posição do braço. Para os testes, os comandos foram gerados à partir de uma placa Arduino Mega 2560(\autoref{fig_garra}).\par

	\begin{figure}[h]
	\caption{\label{fig_garra} Garra do braço robótico }
	\begin{center}
		\includegraphics[width=50mm,scale=0.5]{IMG_20160523_203903430.jpg}	
	\end{center}
	\legend{Fonte: Autor}
\end{figure}

	\subsection{Modelagem de movimentação do braço}\label{subsec-modelagem}
	Foi feita a modelagem de todos os movimentos possíveis para o braço robótico para posterior uso como referência durante o projeto.\par
	\subsubsection{Modelagem inicial}\label{subsubsec-modelageminic}
	Inicialmente, foi feita a modelagem para dois motores (ombro e cotovelo), e gradualmente o modelo foi melhorado(\autoref{fig-mov2D}). \par
	\begin{figure}[!htb]
	\caption{\label{fig-mov2D} Coordenadas X-Y para todos os valores de theta1 e theta2 }
	\begin{center}
		\includegraphics[width=100mm]{2dof.jpg}	
	\end{center}
	\legend{Fonte: Autor}
	\end{figure}

	\subsubsection{Modelagem da movimentação do braço para os motores correspondentes ao ombro, cotovelo e rotação do ombro do braço}\label{subsubsec-modelagem3m}

	O modelo foi ampliado para incluir o terceiro dos 4 motores de movimento do braço(\autoref{fig-mov3D}).\par
	\begin{figure} [!htb]
	\caption{\label{fig-mov3D}  Coordenadas X-Y-Z para todos os valores de theta1, theta2 e theta3 }
	\begin{center}
		\includegraphics[width=100mm]{3dof.jpg}	
	\end{center}
	\legend{Fonte: Autor}
	\end{figure}

	\subsubsection{Modelagem da movimentação do braço para os motores correspondentes ao ombro, cotovelo, rotação do ombro e pulso do braço}\label{subsubsec-modelagem4m}

	Inclusão do quarto e último motor referente ao braço, completando o modelo a ser utilizado(\autoref{fig-mov4D}).\par

	\begin{figure}[htb!]
	\caption{\label{fig-mov4D}  Coordenadas X-Y-Z para todos os valores de theta1, theta2, theta3 e theta4 }
	\begin{center}
		\includegraphics[width=100mm]{4dof.jpg}	
	\end{center}
	\legend{Fonte: Autor}
\end{figure}

\subsection{Aquisição de imagens pela camera}\label{subsec-cameras}

Figura 17: Imagem das cameras estereoscópicas
A aquisição das imagens foi feita usando um computador Raspberry Pi rodando o sistema operacional Raspbian Jessie.\par 

Para isso, foram testados os programas motion, gstreamer e Mjpg-Streamer. A seguir,  é feita uma comparação dos programas:
motion: baixa complexidade de configuração, taxa de captura baixa, alto consumo de capacidade de processamento. Foi logo descartado
Gstreamer: alta complexidade para configuração inicial, taxa de captura moderada, alto consumo de capacidade de processamento, não amigável ao uso de duas câmeras.\par
mjpg-Streamer: Configuração inicial de dificuldade moderada, taxa de captura adequada quando configurado para modo de baixa resolução.
Após a escolha do programa, foi feito o estudo para a captura em tempo real de duas câmeras ao mesmo tempo. Para isso, foi primeiro feito o teste para duas instâncias do programa rodando ao mesmo tempo, após ajuste de taxa de captura e resolução provou-se que o programa mantinha-se estável o suficiente para aplicação. (\autoref{fig_cameras}).\par

\begin{figure}[h!]
	\caption{\label{fig_cameras}  Imagem obtida pelas câmeras estereoscópicas usando mjpg-streamer}
	\begin{center}
		\includegraphics[width=100mm]{13898174_1266308583399952_1266040180_o.png}	
	\end{center}
	\legend{Fonte: Autor}
\end{figure}




\subsection{Montagem do sistema de motores e controle dos motores da câmera}\label{subsec-motores-camera}

Após a configuração das câmeras, foi feita a montagem de sua base com motores, que pode ser vista abaixo(\autoref{fig_camera_mount}): 

\begin{figure}[h!]
	\caption{\label{fig_camera_mount}  Montagem das câmeras estereoscópicas}
	\begin{center}
		\includegraphics[width=100mm]{14407821_1315291101835033_240409903_o.jpg}	
	\end{center}
	\legend{Fonte: Autor}
\end{figure}

Após a montagem, foi feita a configuração de duas placas Particle Photon que foram utilizadas como transmissor e receptor, devido ao seu tamanho e peso reduzidos e ao fato de possuir uma antena WiFi já embutida.
Para a captura dos movimentos da cabeça do usuário, decidiu-se acoplar um giroscópio externo aos óculos de realidade virtual, de modo que o projeto não dependesse de formas de captura de movimentos específicas para cada modelo de óculos.\par
\begin{figure}[h!]
	\caption{\label{fig_vrglasses}  Óculos de realidade virtual com destaque para o giroscópio externo}
	\begin{center}
		\includegraphics[width=100mm]{14424070_1315290465168430_612088294_o.jpg}	
	\end{center}
	\legend{Fonte: Autor}
\end{figure}
Para o transmissor, foram realizadas duas versões de código.
Inicialmente, foi  escrito um código usando-se a saída do giroscópio e realizando as conversão usando-se como base o algoritmo descrito em \cite{debra}. Este código era altamente dependente da posição inicial do giroscópio, que era usada para uma rotina de calibragem inicial.\par

A segunda versão do código foi realizada usando-se bibliotecas específicas ("MPU6050/I2Cdev.h" e "MPU6050/MPU6050\_6Axis\_MotionApps20.h") traduzidas de arduino, que fazem uso de algoritmos internos ao giroscópio para a calibragem , tornando-a mais rápida e precisa \par

Para o receptor, foi feita uma única versão, responsável por receber os ângulos de rotação e transmiti-los aos dois motores na base da câmera 
A transmissão dos dados atualmente se dá via protocolo UDP em uma rede local.\par

\subsection{ Captura de ângulos usando Processing}\label{subsec-Processing}
\begin{figure}[h!]
	\caption{\label{fig_processing}  Imagem obtida pelas câmeras estereoscópicas}
	\begin{center}
		\includegraphics[width=100mm]{processing.png}	
	\end{center}
	\legend{Fonte: Autor}
\end{figure}
Foi utilizado inicialmente a linguagem Processing (baseada em Java), usando-se da bibliotenca OpenNI para a captura dos movimentos. No entanto, devido a baixa precisão da captura usando a biblioteca, essa abordagem foi abortada.\par


\subsection{ Captura de ângulos usando Visual C\#}\label{subsec-visualCsharp}
A programação foi realizada em Visual C\# usando-se como base o exemplo de captura e visualização de esqueleto do Microsoft Kinect 1.8 fornecido pela Microsoft, ao qual foram adicionadas funcionalidades necessárias ao projeto. No estado atual, o programa pode além de exibir o esqueleto com as juntas do usuário, capturar os ângulos do braço do usuário e exibi-los como números, para que seja feito o teste de coerência dos dados a serem enviados(\autoref{fig_visualcsharp}).\par
\begin{figure}[h!]
	\caption{\label{fig_visualcsharp}  Captura de ângulos do braço humano usando Visual C\# e Kinect}
	\begin{center}
		\includegraphics[width=100mm]{vbscreen.png}	
	\end{center}
	\legend{Fonte: Autor}
\end{figure}


\subsection{Controle dos servomotores via conexão serial usando Visual C\#}\label{subsec-VisualCsharpcontrol}
Foi feita a integração do controle dos servomotores com o programa responsável pela captura dos movimentos do usuário via conexão serial USB(\autoref{fig_base}). \par
\begin{figure}[h!]
	\caption{\label{fig_base}  Teste do motor da base do braço}
	\begin{center}
		\includegraphics[width=100mm]{20161101_110237.jpg}	
	\end{center}
	\legend{Fonte: Autor}
\end{figure}
\subsection{Controle dos servomotores via conexão UDP usando Visual C\# e testes de integração}\label{subsec-all}
Nesta última etapa, foi feita a tradução do código para a SDK 2.0 do Kinect e a inclusão da funcionalidade WiFi ao programa. Com isso, foi feito de todo o sistema o teste em rede local, em que os atrasos de transmissão foram considerados irrelevantes (<0.1s)(\autoref{fig_integra;:ao}).\par
		\begin{figure}[h!]
			\caption{\label{fig_integrated}  Teste de integração}
			\begin{center}
				\includegraphics[width=100mm]{integrated.png}	
			\end{center}
			\legend{Fonte: Autor}
		\end{figure}

\section{Resultados}\label{sec-resultados}	
À partir da execução do projeto, foi possivel concluir que existe um grande espaço para o desenvolvimento deste tipo de projeto em aplicações futuras.\par
Foi constatado que o sistema de câmeras, apesar da baixa resolução do hardware utilizado, é capaz de fornecer uma razoável noção de profundidade, e aliado com a sua capacidade de seguir a movimentação do usuário, tem aplicações comerciais imediatas.\par
Percebeu-se no entanto, que existe uma baixa resistência a ruídos por parte do sensor óptico Microsoft Kinect que apresentava altas taxas de erro de leitura em testes com mais de um ser humano em tela. \par
 É relevante notar também que o mesmo apresentou dificuldade na leitura do usúario quando este se encontrava de perfil em relação ao sensor, o que, aliado a falta de noção espacial do usuário em relação ao próprio corpo quando no modo de imersão, pode causar dificuldades na interação com o ambiente remoto, com o possível descontrole do braço.\par
 É possível assim, concluir que ainda são necessárias melhorias antes de tal abordagem ser aplicável a usos comerciais.
	% ----------------------------------------------------------

	% ----------------------------------------------------------
	% Prepara pdf para iniciar o bookmark na raiz
	% ----------------------------------------------------------
	\phantompart
		% ----------------------------------------------------------

	% ----------------------------------------------------------
	% Conclusão
	% ----------------------------------------------------------
	% ---
% Arquivo com a conclusão do Trabalho de Conclusão de Curso dos alunos
% Daniel Noriaki Kurosawa
% da Escola Politécnica da Universidade de São Paulo
% ---
	\chapter{Conclusão}\label{cap-conclusao}
	
	O sistema desenvolvido permitiu a interação do usuário com um ambiente através da coleta de dados de movimento tanto por tecnologia vestível como por tecnologia sem marcadores e a imersão visual estereoscópica neste ambiente, cumprindo com grande parte das especificações do projeto.\par

	Entretanto, o escopo do trabalho proposto inicialmente foi reduzido para entrega
	deste documento. Assim, os testes que antes seriam realizados com o sistema conectado a internet foram executados somente em rede local. \par
	
	Isto pode ser explicado pelos desafios enfrentados durante o desenvolvimento do
	projeto, tais como:
		\begin{itemize}
	 \item Utilização de tecnologias diferentes: cada componente foi desenvolvido em plataformas e linguagens distintas, o que demandou um considerável tempo de familiarização e estudo. Além disso, houveram mudanças de hardware durante o projeto, o que exigiu que alguns códigos tivessem que ser mudados
	 
	 \item Integração dos componentes do sistema: a comunicação entre os componentes, embora utilize técnicas e protocolos existentes, foi consideravelmente trabalhosa e de difícil depuração.
	
	
	\item Ausência de certos componentes no mercado brasileiro: Parte do hardware teve que ser importada ou construída especialmente para o projeto, tirando o foco temporariamente da programação e causando atrasos generalizados no projeto.
	\end{itemize}
	
	Algumas sugestões de trabalhos futuros para aprimoramento deste projeto são:
	\begin{itemize}
	\item Identificação e seleção do usuário a operar o sistema;
	
	\item Gravação dos dados de movimento do usuário para execução de tarefas repetitivas;
	
	\item  Uso de motores com feedback da posição atual para medida de precisão e implementação de algoritmos de controle;
	
	 \item Armazenamento das imagens obtidas pelas câmeras;
	 \item Adaptação do código a outros tipos de atuadores;
	
	 \item Adição de um sistema de locomoção para o sistema;
	 \item Configuração de um servidor conectado à internet para aumento da distância de operação;
	 \item Adição de conexão GSM, para controle independente de roteadores locais.
	 \item  Melhorias de hardware/software para melhora de precisão;
	 
	\end{itemize}

	Finalmente, foi possivel concluir que existe uma grande possibilidade no desenvolvimento para o método de operação de braço robótico estudado em aplicações futuras, mas que no entanto, no estado atual da tecnologia, usos comerciais ainda se provam inviáveis devido a alta incidência de erros de leitura.Foi possível concluir também, que a estereoscopia pode ter aplicações imediatas, com grandes possibilidades de desenvolvimento para um futuro próximo\par
	
	\section*{Comentários finais}\label{sec-comentarios}
	Devido ao caráter interdisciplinar deste projeto, não só foi possível como necessária a aquisição de muitos conhecimentos, técnicas e termos da engenharia de computação, sendo todo o processo do projeto marcado pelo aprendizado.\par
	
	Tal característica do projeto reflete o ideal engenheiro de busca pelo conhecimento que permeia todo o curso na Escola Politécnica.\par
	
	Finalmente, julgo como satisfatória e gratificante a experiência de poder realizar todas as etapas de um projeto de engenharia, desde sua concepção até sua conclusão e análise dos resultados, podendo observar em primeiro plano a execução e aplicação de conceitos teóricos vistos (e não vistos) em aula. 
	
	
	% ----------------------------------------------------------
	
	% ----------------------------------------------------------
	% ELEMENTOS PÓS-TEXTUAIS
	% ----------------------------------------------------------
	\postextual
	% ----------------------------------------------------------
	
	% ----------------------------------------------------------
	% Referências bibliográficas
	% ----------------------------------------------------------
	\bibliography{referencias-arm}
	
	% ----------------------------------------------------------
	% Glossário
	% ----------------------------------------------------------
	%\glossary
	% ----------------------------------------------------------
	
	% ----------------------------------------------------------
	% Apêndices - SEM APENDICES
	% ----------------------------------------------------------
	%\begin{apendicesenv}
		%\partapendices % pagina indicando inicio dos apendices
		%\chapter{Quisque libero justo}
		%\lipsum[50]
	%\end{apendicesenv}
	% ----------------------------------------------------------
	
	% ----------------------------------------------------------
	% Anexos - SEM ANEXOS
	% ----------------------------------------------------------
	%\begin{anexosenv}
		%\partanexos	% Imprime uma página indicando o início dos anexos
		%\chapter{Morbi ultrices rutrum lorem.}
		%\lipsum[30]
	%\end{anexosenv}
	% ----------------------------------------------------------
	
	%---------------------------------------------------------------------
	% INDICE REMISSIVO
	%---------------------------------------------------------------------
	\phantompart
	\printindex
	%---------------------------------------------------------------------
	
\end{document}